\documentclass[a4paper, 10pt]{article}
\usepackage{pgf}
\usepackage{eurosym}
\usepackage{graphicx}
\usepackage{wasysym}
\usepackage{hyperref}
\usepackage{listings}
\usepackage{pxfonts}
\usepackage{verbatim}
\usepackage{color}
\usepackage{xcolor}
\usepackage{wrapfig}
\usepackage{enumitem}
\usepackage{booktabs}
\usepackage{tabularx}

\hypersetup{
    bookmarks=true,         % show bookmarks bar?
    unicode=true,          % non-Latin characters in Acrobat’s bookmarks
    pdftoolbar=true,        % show Acrobat’s toolbar?
    pdfmenubar=true,        % show Acrobat’s menu?
    pdffitwindow=true,     % window fit to page when opened
    pdftitle={Assignment 2},    % title
    pdfauthor={Paul Vesey},     % author
    pdfsubject={Construction Project Management},   % subject of the document
    pdfcreator={},   % creator of the document
    pdfproducer={xelatex}, % producer of the document
    pdfkeywords={'Project Management' }, % list of keywords
    pdfnewwindow=true,      % links in new PDF window
    colorlinks=true,       % false: boxed links; true: colored links
    linkcolor=violet,          % color of internal links (change box color with linkbordercolor)
    citecolor=magenta,        % color of links to bibliography
    filecolor=red,      % color of file links
    urlcolor=blue           % color of external links
}

\setlength\parindent{0pt}
\begin{document}

\lstset{language=HTML,
				basicstyle=\small,
				breaklines=true,
        numbers=left,
        numberstyle=\tiny,
        showstringspaces=false,
        aboveskip=-20pt,
        frame=leftline
        }
				
\begin{table}%
	\begin{minipage}{0.4\textwidth}%
			\includegraphics[width=1\textwidth]{../img/LITlogo.jpg}
	\end{minipage}
	\qquad
	\centering
	\parbox{0.4\textwidth}{
		\begin{large}			
			\begin{tabular}{| r | l |} \hline
				Subject: & \textbf{Construction Project Management}\\
				Course: & \textbf{CPM Special Purpose Award}\\	
				Session: & \textbf{Autumn 2019}\\
				Lecturer: & \textbf{Paul Vesey \footnotesize{BEng, MIE, HDip}}\\
				\hline
			\end{tabular}
		\end{large}			
	}
\end{table}
\vspace{0.25cm}	
	

\part*{Assignment 2 (35\%)- Autodesk Construction Cloud Group Project}


\begin{tabularx}{\textwidth}{ |X|X| }
	\hline
	\textbf{Issue Date:} & 11$^{th}$ November 2021\\
	\hline 
	\textbf{Submission Date:}  & 5$^{th}$ March 2022\\
	\hline
\end{tabularx}

\section*{Introduction}

This assignment will examine the following learning outcomes:\\

\begin{tabularx}{\textwidth}{ |c|X|c| }
	\hline
	\textbf{No.} & \textbf{Learning Outcome} & \textbf{Assessed} \\
	\hline 
	1  & Function as a project manager on a civil engineering project from project inception to completion. & Yes \\
	2  & Manage a project team and evaluate team performance. & Yes \\
	3  & Demonstrate and apply the concepts, techniques and skills in all ten knowledge areas of project management to construction and civil engineering projects. & Yes \\
	4  & Assess project risk and take appropriate mitigation measures. & No \\
	5  & Demonstrate skill and competence with current generation BIM project management tools. & Yes \\
	\hline
\end{tabularx}


In this assignment you are required to utilise Autodesk Construction Cloud (formally BIM360) for a number of tasks on a Model/Project that has been provided to you.  You should examine these documents in detail to gain an understanding of the project.\\

\begin{enumerate}
	\item Document Management 
	\item Project Management
	\item Model Coordination
	\item Field Management
	\item Data Analytics
\end{enumerate}

You have been added to an Autodesk Construction Cloud (ACC) project entitled 'PM \& BIM Assignment 2 - 2021-2022'.  A screenshot of a portion of the model is shown in figure \ref{fig:thumbnail}.  ACC can be accessed at this link - \href{https://acc.autodesk.com/}{acc.autodesk.com/}

\begin{figure}
	\centering
	\includegraphics[width=0.7\linewidth]{RevitAssets/Thumbnail}
	\caption[Section Box Image of Model]{Section Box Image of Model}
	\label{fig:thumbnail}
\end{figure}





\section*{Scenario}



This is a project for 4 individuals. All team members are expected to contribute equally to the project. As a team you are responsible for managing the workload and division of responsibility. All team members will receive the same mark, based on the academic merit of the complete submission.\\


In general, one Team member should implement an item or element of functionality, while each of the remaining 3 team members either test or respond as appropriate.  Your report should detail how tasks were apportioned in an assignment matrix similar to the one shown in Table \ref{tab:AM}

\begin{table}[ht]
	\centering
	\begin{tabular}{|l|c|c|c|c|}
		\hline
		& \textbf{Name A} & \textbf{Name B} & \textbf{Name C} & \textbf{Name D} \\
		\hline
		\textbf{Task 1} & Implemented & Tested & Tested & Tested \\
		\hline
		\textbf{Task 2} & Responded & Implemented & Responded & Responded \\
		\hline
		\textbf{Task 3} & Responded & Responded & Responded & Implemented \\
		\hline
	
	\end{tabular}
	\caption{Sample Table}
	\label{tab:AM}
\end{table}





\newpage

\section{Document Management}

\begin{figure}[h!t]
	\includegraphics[width=4.0cm]{RevitAssets/docs}
	\label{fig:docmgmt}
\end{figure}


\subsection{Issue}

Each team member should create one valid issues on the any of the models/views in the project; 4 issues in total.  The issue creator should assign at least one other team member to action and close the issue.  Document each of these issues in your report, including the Issue ID.

\subsection{Approval Workflow}

One team member should create an Approval Workflow that involves 2 steps.  The fist step should be completed by two team members, and the second step by the other two team members.  Document the configuration settings in your report; an annotated screenshot is suitable for the report.  Name your review 'Team XX - Approval'  

\subsection{Transmittal}

Create a Transmittal for three documents from the project.  One team member should create the transmittal and the others should be the recipients.  Also include \href{mailto:paul.vesey@tus.ie}{paul.vesey@tus.ie} as a recipient.  Name your Transmittal 'Team XX - Transmittal'

\subsection{Table for Report}

\begin{table}[ht]
	\centering
	\begin{tabular}{|l|c|c|c|c|l|}
		\hline
		& \textbf{Name A} & \textbf{Name B} & \textbf{Name C} & \textbf{Name D} & \textbf{ACC ID}\\
		\hline
		\textbf{Task 1.1-a} & Creator & Responder & Responder & Responder & \#code\\
		\textbf{Task 1.1-b} & Responder & Creator & Responder & Responder & \#code\\
		\textbf{Task 1.1-c} & Responder & Responder & Creator & Responder & \#code\\
		\textbf{Task 1.1-d} & Responder & Responder & Responder & Creator & \#code\\
		\hline
		\textbf{Task 1.2.1} & Workflow Creator &  &  &  & \#code\\
		\textbf{Task 1.2.2} & Stage 1 & Stage 1 & Stage 2 & Stage 2 & \\
		\hline
		\textbf{Task 1.3} & Creator & Recipient & Recipient & Recipient & \#code\\
		\hline
	\end{tabular}
	\caption{Document Report Table }
	\label{tab:docs-items}
\end{table}





\newpage

\section{Build - RFI}

\begin{figure}[h!t]
	\includegraphics[width=4.0cm]{RevitAssets/rfi}
	\label{fig:projmgmt}
\end{figure}



\subsection{RFI Creation}
The default RFI Workflow will be used for this project.  Each team member should create an RFI, leading to 4 RFIs associated with the team.  All RFI's should be responded to in full by the other members.  Document each of these RFIs within your report. Annotated screenshots are suitable for the report.  Please note that the RFI functionality is no longer attached to the model assets.  RFIs are now stand-alone entities that may or may not contain references to a model.

\subsection{Table for Report}

\begin{table}[ht]
	\centering
	\begin{tabular}{|l|c|c|c|c|l|}
		\hline
		& \textbf{Name A} & \textbf{Name B} & \textbf{Name C} & \textbf{Name D} & \textbf{ACC ID}\\
		\hline
		\textbf{Task 2.1-a} & Creator & Responder & Responder & Responder & \#code\\
		\textbf{Task 2.1-b} & Responder & Creator & Responder & Responder & \#code\\
		\textbf{Task 2.1-c} & Responder & Responder & Creator & Responder & \#code\\
		\textbf{Task 2.1-d} & Responder & Responder & Responder & Creator & \#code\\
		\hline
	\end{tabular}
	\caption{RFI Report Table }
	\label{tab:rfi-items}
\end{table}





\newpage

\section{Model Coordination}
\begin{figure}[h!t]
	\includegraphics[width=4.0cm]{RevitAssets/modelcoord}
	\label{fig:modelmgmt}
\end{figure}


\subsection{Coordination Space}
Create one Coordination Space for the team, via the Model Coordination Settings controls.  Your team name should be included in the coordination space name, for instance 'Skanska Coordination Space'.  Document the details of this space in your report.
\subsection{Clash Issue Creation}
Create 4 valid clash issues from the Coordination Space created in the previous sub-section.  Each of these issues should be Assigned, Actioned and Closed.  Document each of these issues in your report.

\subsection{Table for Report}

\begin{table}[ht]
	\centering
	\begin{tabular}{|l|c|c|c|c|l|}
		\hline
		& \textbf{Name A} & \textbf{Name B} & \textbf{Name C} & \textbf{Name D} & \textbf{ACC ID}\\
		\hline
		\textbf{Task 3.1} & Clash View Creator &  &  &  & \\
		\hline
		\textbf{Task 3.2-a} & Creator & Responder & Responder & Responder & \#code\\
		\textbf{Task 3.2-b} & Responder & Creator & Responder & Responder & \#code\\
		\textbf{Task 3.2-c} & Responder & Responder & Creator & Responder & \#code\\
		\textbf{Task 3.2-d} & Responder & Responder & Responder & Creator & \#code\\
		\hline
	\end{tabular}
	\caption{Clash Report Table }
	\label{tab:clash-items}
\end{table}


\newpage
\section{Asset Management}

\begin{figure}[h!t]
	\includegraphics[width=4.0cm]{RevitAssets/assets}
	\label{fig:assets}
\end{figure}

\subsection{Asset Creation Creation}
\label{section:assetcreation}
Each member of the team is to create four (4) assets on ACC based on the Revit Models.  Asset categories and sub-categories have been setup to assist in this task.  Each asset record is to include an appropriate specification document obtained online from a suitable manufacturer.  The location field is to be populated with either 'Ground Floor', 'First Floor' or 'Second Floor'.


\subsection{Asset Barcode Creation}
The asset pack for this assignment contains a pdf of Type 39 Barcodes.  Three (3) barcodes are provided for each K-number.  These barcodes are to be added to three (3) of the asset records created in \ref{section:assetcreation}.  These are then to be tested using the ACC field app, Plangrid Build.  Plangrid Build is available for android from\\ \href{https://play.google.com/store/apps/details?id=com.plangrid.android&hl=en_IE&gl=US}{https://play.google.com/store/apps/details?id=com.plangrid.android\&hl=en\_IE\&gl=US} \\and for iOS at\\ \href{https://apps.apple.com/us/app/plangrid-build-field-app/id498795789}{https://apps.apple.com/us/app/plangrid-build-field-app/id498795789}\\
Your report should include screen-captures of the asset accessed through via the barcodes.

\subsection{Asset QR-Code Creation}
The asset pack for this assignment contains a qr-code in png format for each K-number.  The code is to be added to one of the assets created in \ref{section:assetcreation}.

Your report should include screen-captures of each of the asset accessed through the qr-code.

\newpage
\subsection{Table for Report}

\begin{table}[ht]
	\centering
	\begin{tabular}{|l|c|c|c|c|}
		\hline
		& \textbf{Name A} & \textbf{Name B} & \textbf{Name C} & \textbf{Name D} \\
		\hline
		\textbf{Task 4.1-a} & Asset Name & Asset Name & Asset Name & Asset Name\\
		\textbf{Task 4.1-b} & Asset Name & Asset Name & Asset Name & Asset Name\\
		\textbf{Task 4.1-c} & Asset Name & Asset Name & Asset Name & Asset Name\\
		\textbf{Task 4.1-d} & Asset Name & Asset Name & Asset Name & Asset Name\\
		\hline
		\textbf{Task 4.2-a} & Barcode & Barcode & Barcode & Barcode\\
		\textbf{Task 4.2-b} & Barcode & Barcode & Barcode & Barcode\\
		\textbf{Task 4.2-c} & Barcode & Barcode & Barcode & Barcode\\
		\textbf{Task 4.2-d} & Barcode & Barcode & Barcode & Barcode\\
		\hline	
		\textbf{Task 4.3-a} & QR-Code & QR-Code & QR-Code & QR-Code\\
		\hline
	\end{tabular}
	\caption{Assets Report Table }
	\label{tab:asset-items}
\end{table}


\newpage
\section{Forms Functionality}

\begin{figure}[h!t]
	\includegraphics[width=4.0cm]{RevitAssets/forms}
	\label{fig:dieldmgmt}
\end{figure}



\subsection{Template Creation}

Each team member is to create a checklist template on a domain/discipline of their own choice.  The domains/disciplines must not be duplicated or overlapped, and therefore the team must coordinate their work.  Each template must contain at least 3 sections with 5 questions/tests in each section.  Document each template in your report.

\subsection{Checklist Creation}

Each team member is to complete 3 checklists for testing purposes.  The findings of these tests is to be fed back to the template creator for their comments or to make amendments.  These correspondences should be fully recorded and tracked using the Issue Tracking System in ACC. These issues should have the 'Project Admin' category, and 'Checklist Testing' type set in the 'Type' field of the issue creation dialogue.  The creation/testing process is to be documented within your report.  

\subsection{Table for Report}

\begin{table}[ht]
	\centering
	\begin{tabular}{|l|c|c|c|c|l|}
		\hline
		& \textbf{Name A} & \textbf{Name B} & \textbf{Name C} & \textbf{Name D} & \textbf{ACC ID}\\
		\hline
		\textbf{Task 5.1-a} &  Creator &  &  &  & \#code\\
		\textbf{Task 5.1-b} &  & Creator &  &  & \#code\\
		\textbf{Task 5.1-c} &  &  & Creator &  & \#code\\
		\textbf{Task 5.1-d} &  &  &  & Creator & \#code\\
		\hline
		\textbf{Task 5.2-a} & - & Test & Test & Test & \#code\\
		\textbf{Task 5.2-b} & Test & - & Test & Test & \#code\\
		\textbf{Task 5.2-c} & Test & Test & - & Test & \#code\\
		\textbf{Task 5.2-d} & Test & Test & Test & - & \#code\\
		\hline
	\end{tabular}
	\caption{Forms Report Table }
	\label{tab:field-items}
\end{table}


\newpage

\section{Data Analytics - Insight}

\begin{figure}[h!t]
	\includegraphics[width=4.0cm]{RevitAssets/insight}
	\label{fig:insight}
\end{figure}



Create the following standard reports using ACC Insight: 

\begin{table}[ht]
	\centering
	\begin{tabular}{|p{0.2\textwidth}|p{0.3\textwidth}|p{0.3\textwidth}|p{0.1\textwidth}|}
		\hline
		\textbf{Name} & \textbf{Filter 1} & \textbf{Filter 2} & \textbf{ACC ID}\\
		\hline
		\textbf{Coordination Issue Detail} & \textit{Type:} Coordination & Created by Team Members & \#code\\
		\hline
		\textbf{Coordination RFI Detail} & \textit{Category:} Design Coordination & Created by Team Members & \#code\\
		\hline
		\textbf{Issue Summary} & Created by Team Members &  & \#code\\
		\hline
		\textbf{RFI Summary} & Created by Team Members & All Status Types & \#code\\
		\hline
	\end{tabular}
	\caption{Insight Report Table }
	\label{tab:insight-items}
\end{table}



Each team member is responsible for creating one 'Insight' report.  All reports should be filtered so that only data 'assigned to' or 'created by' the team members is shown.  All reports should be run and exported as .pdf files.  These files are to be included in your report, along with a screenshot of the report configuration as above in ACC.\\

\subsection{Table for Report}

\begin{table}[ht]
	\centering
	\begin{tabular}{|l|c|c|c|c|l|}
		\hline
		& \textbf{Name A} & \textbf{Name B} & \textbf{Name C} & \textbf{Name D} & \textbf{ACC ID}\\
		\hline
		\textbf{Task 6.1-a} &  Creator &  &  &  & \#code\\
		\textbf{Task 6.1-b} &  & Creator &  &  & \#code\\
		\textbf{Task 6.1-c} &  &  & Creator &  & \#code\\
		\textbf{Task 6.1-d} &  &  &  & Creator & \#code\\
		\hline
	\end{tabular}
	\caption{Insight Report Table }
	\label{tab:report-items}
\end{table}



\newpage
\section{Report}
A single report of 4000 words should be submitted by each team.  Your report should be laid out in a professional manner, as would be done for a client or employer.  This involves, \emph{inter-alia}, a cover page, table of contents, and logical sectioning.  You report must include, but is not limited to:
\begin{itemize}
	\item Details of the methodology used to create each component
	\item Screen captures and images of system functionality
	\item Details of any challenges that you experienced with the application
	\item Commentary on learning experience and the skills you developed during the production of this assignment.
\end{itemize}
  


\section{Submission}
The bulk of the deliverables will be created and delivered on Autodesk Construction Cloud.  However, all computer files and the written report are to be zipped into a single file and uploaded to Microsoft Teams on or before the date and time indicated.  Key components of the submission are:

\begin{itemize}
	\item Autodesk Construction Cloud work as described
	\item Report on the methodology used to meet the work requirements.  
\end{itemize}



\subsection*{Late Submission}
Failure to submit your assignment on or before the date and time indicated on Moodle will result in a penalty of 5\% per day or part thereof.

\subsection*{Marking Scheme}

\begin{table}[h!]
	\begin{center}
	\begin{tabular}{p{5cm}  p{5cm} }
     	\toprule
		\textbf{Element} & \textbf{Proportion} \\ 
    	\cmidrule(r){1-1}\cmidrule(lr){2-2}
      	Document Management & 30\%\\
      	Project Management & 10\%\\
      	Model Coordination & 20\%\\
      	Asset Management & 15\%\\
      	Field Management & 15\%\\
      	Data Analytics & 10\%\\
      	\bottomrule
    \end{tabular}
    \label{tbl:markSchemeAsmt2}
    \end{center}
\end{table}

\end{document}